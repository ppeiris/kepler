\chapter{Conclusion}

%\todo{Conclusion discussion - based on the conclusion section of the paper}
%\todo{Describe the entier process - highlight whats interesting and whats difficult, interesting - Astrophysics can replace the physical data processing piplelines with machine learning and allow real time dataprocessing with the large data comming from the telescopes}
%\todo{How can I Improve - this is where explain we can now do this using light curves instead using TCE catalog}

Machine learning techniques offer a way to automate some stages of the exoplanet discovery. In this project, I have demonstrated the backpropagated Multilayered Perceptron neural network is quite good at distinguishing between systematic noise, eclipsing binary and exoplanet candidates signatures from the TCE catalog. Within Kepler pipeline, there are plenty of statistical analysis methods have been using to scrutinize each light curve that is coming from Kepler spacecraft observation and also many other follow-up observations. Machine learning is something that has not been widely used in astrophysics. My primary goal is to introduce advanced machine learning techniques to analysis this astrophysical dataset that usually takes a much labor intensive process to automate. 

Workflow of this project as follows. NASA has a launch a spacecraft called Kepler in 2009 to observe a particular are of the sky for an extended period of time to understand the abundance of exoplanets, more specifically Earth-like planets. These observations are in the format of light curves. Light curves are build using the flux that coming from the star. These light curves have the flux fluctuations when planets are orbiting in front of the star. The fluctuations include other follow-up observations using many other ground and space-based telescopes. NASA has built a catalog called Threshold Crossing Event catalog (TCE) using this data. This TCE catalog further process using a pipeline called Kepler pipeline. During this pipeline, the data has been reduced to meaningful parameters where we can extract astrophysically interesting information. This catalog is available to for scientists to download for further studies. I have download this entire catalog from the NASA Exoplanet archives for this project. I have process this catalog and use the dimensionality reduction method (Principal component analysis) to reduce the dimensions of the dataset. The reduced data set then split into two sets: training and testing data. 

I have used the scikit learn machine learning package to build a backpropagation Multilayered Neural network (using python) and trained it using the training dataset. During this process, various parameters are tested to build a most appropriate neural network to analysis the TCE catalog. Once we build a well tuned neural network, I process the Kepler TCE catalog. This TCE catalog data is already scrutinized by a large number of scientists and also have done many follow-up observations using other telecopes to confirm the classifications. I was able to build a neural network to make this classification to 99$\%$ accuracy. Searching in the astrophysics literature, this is the first time anybody who used a neural network to make the classification for the Kepler dataset. 


As a next step, I am planning to advance this study further and introduce more machine learning into astrophysics world to accurately and quickly process data. Wtith the advancement of CCD technology and noise reduced electronics, and today telescopes produce much larger datasets within a short period with less noise. These data still being mostly analysis by manually by scientists and graduate students. With the help of machine learning, we will be able to build models where we can process these large datasets and make discoveries in much shorter timescales. It is vital for us to introduce these methods to expedite the data analysis process in astrophysics. Within this project, I am using the TCE catalog which is built by the Kepler pipeline using the light curves. I have proved that a simple neural network can process this dataset more accurately. As a next step, I am planning to advance further this study to build a machine learning algorithm pipeline to analysis the raw light curves that are coming from the spacecraft directly. Kepler mission has been completed; I will be able to use the existing lightcurve data and TCE catalog as my training dataset to build this platform where NASA and other fellow scientists will be able to use it to analysis the data from the future NASA missions that are scheduled to further advance the exoplanet hunts. It will be quite fulfilling see someday this work might help us to find extraterrestrial life. 